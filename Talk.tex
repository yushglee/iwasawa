\documentclass[11pt]{amsart}
\usepackage{mathtools}
\usepackage{amssymb,amsthm,amsmath,bbm,euscript,mathrsfs}
\usepackage{tikz-cd}
\usetikzlibrary{decorations.pathmorphing}
\usepackage{enumitem}
\usepackage{color}
\usepackage[backref = page]{hyperref}
\hypersetup{
linkcolor=blue,
linktocpage,
}
\setcounter{tocdepth}{1}

\setlength{\textwidth}{\paperwidth}
\addtolength{\textwidth}{-2in}
\calclayout



\title{Ribet's converse theorem}
\author{Yu-Sheng Lee}
%newcommand
\newcommand{\Q}{\mathbf{Q}}  %rational
\newcommand{\Z}{\mathbf{Z}}  %integer
\newcommand{\R}{\mathbb{R}}  %real
\newcommand{\C}{\mathbb{C}}  %complex
\newcommand{\A}{\mathbb{A}} %adele
\newcommand{\bs}{\mathcal{S}} %Schwartz-Bruhat
\newcommand{\ir}{\mathcal{O}} %interger ring
\newcommand{\p}{\mathfrak{p}} %prime
\newcommand{\tr}{{\rm tr}} %trace
\newcommand{\I}{\mathbb{I}} %identity function 
\newcommand{\G}{\text{Gal}}

\newcommand{\sel}{\operatorname{Sel}}
\newcommand{\Hom}{\operatorname{Hom}}
\newcommand{\End}{\operatorname{End}}
\newcommand{\Ext}{\operatorname{Ext}}
\newcommand{\fitt}{\operatorname{Fitt}}
\newcommand{\Rep}{\operatorname{Rep}}
\newcommand{\Gr}{\operatorname{Gr}}
\newcommand{\gr}{\operatorname{gr}}
\newcommand{\gl}{\operatorname{GL}}
\newcommand{\coker}{\operatorname{coker}}
\newcommand{\Aut}{\operatorname{Aut}}
\newcommand{\frob}{\operatorname{Frob}}
\newcommand{\totfrac}{\operatorname{Frac}}
\newcommand{\trace}{\operatorname{Tr}}
\newcommand{\fil}{\operatorname{Fil}}

% p-adic Hodge
\newcommand{\hot}{\operatorname{HT}}
\newcommand{\dht}{D_{\operatorname{HT}}}
\newcommand{\bht}{B_{\operatorname{HT}}}
\newcommand{\dr}{\operatorname{dR}}
\newcommand{\ddr}{D_{\operatorname{dR}}}
\newcommand{\bdr}{B_{\operatorname{dR}}}
\newcommand{\st}{\operatorname{st}}
\newcommand{\dst}{D_{\operatorname{st}}}
\newcommand{\dpst}{D_{\operatorname{pst}}}
\newcommand{\bst}{B_{\operatorname{st}}}
\newcommand{\cris}{\operatorname{cris}}
\newcommand{\dcris}{D_{\operatorname{cris}}}
\newcommand{\bcris}{B_{\operatorname{cris}}}
\newcommand{\un}{\operatorname{un}}
\newcommand{\et}{\operatorname{\acute{e}t}}
\newcommand{\E}{\mathscr{E}} 
\newcommand{\etmod}{\operatorname{\Phi M}^{\et}} 
\newcommand{\filmod}{\operatorname{MF}} 



\newcommand*{\transp}[2][-3mu]{\ensuremath{\mskip1mu\prescript{\smash{\mathrm t\mkern#1}}{}{\mathstrut#2}}}
\newcommand{\mat}[1]{ \begin{pmatrix}
#1
\end{pmatrix} }
\newcommand{\teim}[1]{ \langle
#1
\rangle }
\newcommand{\ddb}[1]{ \llbracket
#1
\rrbracket }

\makeatletter
\newsavebox{\@brx}
\newcommand{\llangle}[1][]{\savebox{\@brx}{\(\m@th{#1\langle}\)}%
  \mathopen{\copy\@brx\kern-0.5\wd\@brx\usebox{\@brx}}}
\newcommand{\rrangle}[1][]{\savebox{\@brx}{\(\m@th{#1\rangle}\)}%
  \mathclose{\copy\@brx\kern-0.5\wd\@brx\usebox{\@brx}}}
\newcommand{\llbracket}[1][]{\savebox{\@brx}{\(\m@th{#1[}\)}%
  \mathopen{\copy\@brx\kern-0.5\wd\@brx\usebox{\@brx}}}
\newcommand{\rrbracket}[1][]{\savebox{\@brx}{\(\m@th{#1]}\)}%
  \mathclose{\copy\@brx\kern-0.5\wd\@brx\usebox{\@brx}}}
\makeatother

%unindexed theorems
\theoremstyle{definition}
\newtheorem{definition}{Definition}[section]
\theoremstyle{definition}
\newtheorem{theorem}[definition]{Theorem}
\theoremstyle{definition}
\newtheorem{proposition}[definition]{Proposition}
\theoremstyle{definition}
\newtheorem{remark}[definition]{Remark}
\theoremstyle{definition}
\newtheorem{example}[definition]{Example}
\theoremstyle{definition}
\newtheorem{lemma}[definition]{Lemma}

\begin{document}

\maketitle


In this note we discuss 
the relation between Eisenstein congruences and Ribet's converse 
to the Herbrand-Ribet theorem.
Following \cite{Sk}, we treat the theorem
as a specialized case of the Iwasawa main conjecture
and emphasize the role of the congruence modules.
Throughout, let $p$ be an odd prime,
$\chi\colon G_\Q\to\Z_p^\times$ be the $p$-th cyclotomic character, 
and 
\begin{equation*}
    \omega=\overline{\chi}\colon G_\Q
    \to Gal(\Q(\mu_p)/\Q)\to
    \mathbb{F}_p^\times \cong \mu_{p-1}
\end{equation*}
be the Teichmuller character.


\section{The Herbrand-Ribet theorem}

Let $A=Cl(\Q(\mu_p))\otimes\Z_p$ be the $p$-primary part of the field $\Q(\mu_p)$. The action 
of $Gal(\Q(\mu_p)/\Q)$ on $A$ gives a decomposition
\begin{equation*}
    A=\bigoplus_{n=0}^{p-2}A_n
\end{equation*}
where the Galois group acts on $A_n$ by $\omega^n$.
Recall that $p$ is said to be irregular if $A\neq 0$,
and we are interested in the finer problem of whether $A_n\neq 0$.

When $n$ is even,
the Kummer-Vandiver conjectuere states that $A_n$ should be trivial.
Using the Stickelberger elements, Herbrand has shown that $A_0=A_1=0$, and
if $A_{p-k}\neq 0$ for some even $k<p-1$, then $p$ divides the $k$-th Bernoulli number $B_k$
defined by
\begin{equation*}
    \frac{t}{e^t-1}=\sum_{n=0}^\infty B_n\frac{t^n}{n!}.
\end{equation*}
We refer to \cite{Wa} for a detailed discussion of these facts.
On the other hand, Ribet has shown that the converse of Herbrand's result is also true.
\begin{theorem}\cite{Ri}
    Let $2\leq k< p-1$ be an even number. If $p\mid B_k$, then
    $A_{p-k}\neq 0$.
\end{theorem}
\begin{remark}
   By the von Staudt theorem we have $\text{val}_p(B_{p-1})=-1$. This is compatible with the fact that $A_1=0$.
\end{remark}

We first reinterpret $A_n$ as Selmer groups.
Let $G=G_\Q, H=G_{\Q(\mu_p)}, \Delta=Gal(\Q(\mu_p)/\Q)=G/H$ and $W=\mathbb{F}(\omega^n)$,
define
\begin{equation*}
    H^1_f(\Q,W)\coloneqq
    \ker\bigg(
    H^1(\Q,W)\to\prod_{\ell}
    H^1(I_\ell,W)
    \bigg).
\end{equation*}
Since $p\nmid \#\Delta$, we can stare at the inflation-restriction exact sequence
\begin{equation*}
    \begin{tikzcd}
        H^1(\Delta,W) \arrow[r]\arrow[d,equal]&
        H^1(G,W) \arrow[r,"\sim"]\arrow[d]&
        H^1(H,W)^{\Delta} \arrow[r]\arrow[d]&
        H^2(\Delta,W)\arrow[d,equal]\\
        0&H^1(I_\ell,W) \arrow[r,hookrightarrow]
        &\prod_{v\mid \ell}H^1(I_v,W)& 0
    \end{tikzcd}
\end{equation*}
and the isomorphism $H^1(H,W)^{\Delta}\cong \Hom_{\Delta}(H, W)$,
then realize that
\begin{equation*}
    H^1_f(\Q,W)\cong H^1_f(\Q(\mu_p),W)^\Delta
    =\{ \phi\in \Hom_\Delta(H, W)\mid
    f(I_v)=0 \text{ for all } v\}= \Hom(A_n,\mathbb{F}).
\end{equation*}
To show that $A_n\neq 0$, it suffices to show that 
$H^1_f(\Q,\mathbb{F}(\omega^n))$
is nontrivial, or equivalently, that there exists a non-split extension
of Galois representations
\begin{equation*}
    0 \to \mathbb{F}(\omega^n)\to \overline{\rho} \to \mathbb{F}\to 0
\end{equation*}
that splits everywhere locally. We will find such an extension by studying the congruences between Galois representations of modular forms. 

Consider the classical level $1$ Eisenstein series $E_k$ of weight $k$
\begin{equation*}
    E_k=\frac{\zeta(1-k)}{2}+\sum_{n=1}^\infty \sigma_{k-1}(n)q^n,\quad k\geq 4.
\end{equation*}
The constant terms $\frac{\zeta(1-k)}{2}=\frac{-B_k}{2k}$
have the following congruence relations.
\begin{enumerate}
    \item The values $B_k/k$ lie in $\Z_p$ if and only if $(p-1)\nmid k$.
    \item If $k\equiv k'\not\equiv 0\mod (p-1)$, then
    \begin{equation*}
        \frac{B_k}{k}\equiv \frac{B_{k'}}{k'}\mod p.
    \end{equation*}
\end{enumerate}
It follows that if $k<p-1$ is even and $p\mid B_k$,
the Eisenstein series $E_{k+m(p-1)}$
have $p$-integral Fourier coefficients and the constant term is 
divisible by $p$. We may thus assume $k>4$ and $E_k$
is such an Eisenstein series.
Then there exists $k=4a+6b$
such that
\begin{equation*}
    F \coloneqq 
    E_k-\frac{\zeta(1-k)}{2}(240E_4)^a(-504E_6)^b
\end{equation*}
is a nonzero level $1$ cusp form of weight $k$.
The Fourier coefficients of $F$ are congruent to those of the Eisenstein series
\begin{equation*}
    a_\ell(F) \equiv a_\ell(E_k)=1+\ell^{k-1}\mod p.
\end{equation*}
\begin{example}
    Consider $B_{12}=-691/2730$, then
    $E_{12}-(240E_4)^3$ is a nonzero multiple of the
    Ramanujan $\Delta$ function.
\end{example}

Let $S=S_k(1,\Z_p)$ be the space of cusp forms
and $\mathbb{T}=\mathbb{T}_k(1,\Z_p)\subset \End_{\Z_p}(S)$
be the $\Z_p$-subalgebra generated by the Hecke operators. We have the following facts.
\begin{enumerate}
    \item The Hecke algebra $\mathbb{T}$ is reduced and finite flat over $\Z_p$.
    \item There exists a perfect bilinear pairing
    \begin{equation*}
        \mathbb{T}\times S\to \Z_p,
        \quad (T, f)\mapsto a_1(Tf)
    \end{equation*}
    which identifies $S\cong \Hom(\mathbb{T},\Z_p)$.
    A cusp form $f$ is a Hecke eigenform if and only if it corresponds to a
    homomorphism of $\Z_p$-algebras.
    \item The $\Q_p$-algebra $\mathbb{T}\otimes \Q_p$ is semi-simple and 
    has a decomposition $\mathbb{T}\otimes \Q_p=\prod_\lambda K_\lambda$
    into finite extensions over $\Q_p$
    There is a correspondence between the fields (or the minimal primes of $\mathbb{T}$)
    and the conjugacy classes of Hecke eigenforms.
\end{enumerate}

Since $E_k$ is a Hecke eigenform,
the $\Z_p$-module homomorphism associated to $F$
\begin{equation*}
    \eta \colon \mathbb{T}\to \mathbb{F}_p,\quad
    T_\ell\mapsto a_\ell(F)\equiv 1+\ell^{k-1} \mod p
\end{equation*}
becomes a ring homomorphism after modulo by $p$.

\begin{lemma}[Deligne-Serre lifting]
    There exists an eigenform $f\in S_k(1,\ir)$,
    where $\ir$ is the ring of integer of a finite extension over $\Q_p$
    with a uniformizer $\varpi$, such that
    \begin{equation*}
        a_\ell(f) \equiv 1+\ell^{k-1} \mod (\varpi).
    \end{equation*}
\end{lemma}
\begin{proof}
    We apply the going-down to the maximal ideal 
    $\mathfrak{m}\coloneqq\ker(\eta)$ in $\mathbb{T}$.
    Since $\mathfrak{m}\cap \Z_p=(p)$,
    there exists a minimal prime ideal
    $\mathfrak{p}\subset \mathbb{T}$ such that $\mathfrak{p}\cap \Z_p=(0)$.
    The quotient $\mathbb{T}/\mathfrak{p}$ is isomorphic to a ring $\ir$
    as in the statement. Let $f$ be the eigenform corresponding
    to the ring homomorphism $\mathbb{T}\to \mathbb{T}/\mathfrak{p}\cong \ir$.
    The congruence property follows from that $(\varpi)$ pullbacks to $\mathfrak{m}$.
\end{proof}

The irreducible Galois representation associated to $f$
\begin{equation*}
    \rho_f\colon G_\Q\to \gl_2(K),\quad K=\totfrac \ir
\end{equation*}
is unramified away from $p$ and satisfies
$\tr\rho_f(\frob_\ell)=a_\ell(f)$ for all $\ell\neq p$.
It follows from
\begin{equation}\label{cong}
    \tr\rho_f(\frob_\ell)=a_\ell(f)\equiv
    1+\ell^{k-1}=1+\chi^{k-1}(\frob_\ell)\mod (\varpi)
\end{equation}
and the Chebotarev's density that
$\tr\rho_f(\sigma)\equiv 1+\chi^{k-1}(\sigma)\mod (\varpi)$ for all $\sigma\in G_\Q$.
Therefore, the semi-simplified reduction $\overline{\rho}^{ss}$
is isomorphic to $\omega^{k-1}\oplus 1$. To find a lattice whose reduction
gives the desired non-split extension (up to a twist),
we need to employ Urban's lattice construction.
\begin{proposition}[\cite{Ur}]
   Let $\rho\colon G_\Q\to \gl_2(K)$
   be an irreducible Galois representation such that
   \begin{equation*}
       \tr\rho\equiv \chi_1+\chi_2 \mod \mathfrak{a}= (\varpi)^n.
   \end{equation*}
   for characters
    $\chi_i\colon G_\Q\to \ir^\times$ that are distinct modulo $(\varpi)$.
   Then there exists a stable lattice $\mathcal{L}\subset K^2$ whose reduction
   is a non-split extension between
   $\chi_1$ and $\chi_2$ modulo $\mathfrak{a}$.
\end{proposition}
\begin{proof}
    Observe that
    $\det\rho(\sigma)=\tr\rho(\sigma^2)-\tr\rho(\sigma)^2/2$
    and therefore
    \begin{equation*}
        \det(X\mathbf{I}-\rho(\sigma))\equiv (X-\chi_1(\sigma))(X-\chi_2(\sigma))\mod \mathfrak{a}.
    \end{equation*}
    Since $\chi_i$ are distinct modulo $(\varpi)$,
    we can pick $\sigma_0\in G_\Q$ whose characteristic polynomial 
    has distinct roots modulo $(\varpi)$.
    By the Hensel lemma, the the roots lift to distinct eigenvalues in $\ir$.
    We pick a basis $\{v_1,v_2\}$ of eigenvectors and write
    \begin{equation*}
        \rho(\sigma_0)=\mat{\alpha_1&\\&\alpha_2}, \quad
        \alpha_i\in \ir,\quad
        \alpha_i\equiv \chi_i(\sigma_0)\mod \mathfrak{a}.
    \end{equation*}
    Since $\tr\rho(\sigma\sigma_0^n)\equiv \chi_1(\sigma\sigma_0^n)+\chi_2(\sigma\sigma_0^n)\mod \mathfrak{a}$, the relation
    \begin{equation*}
        a_\sigma\alpha_1^n+d_\sigma\alpha_2^n\equiv
        \chi_1(\sigma)\alpha_1^n+\chi_2(\sigma)\alpha_2^n
        \mod \mathfrak{a},\quad
        \rho(\sigma)=\mat{a_\sigma &b_\sigma\\c_\sigma&d_\sigma}
    \end{equation*}
    holds for all $n$ and $\sigma\in G_\Q$. 
    And since $\{(1,1), (\alpha_1,\alpha_2)\}$ generate $\ir^2$, we actually have
    \begin{equation*}
        a_\sigma\equiv \chi_1(\sigma),\quad
        d_\sigma\equiv \chi_2(\sigma) \mod \mathfrak{a}
    \end{equation*}
    for all $\sigma\in G_\Q$ and thus
    \begin{enumerate}
        \item $a_\sigma, d_\sigma \in \ir$ for all $\sigma\in G_\Q$,
        \item $b_\sigma c_\tau=a_{\sigma\tau}-a_\sigma a_\tau\in \ir$
        for all $\sigma, \tau\in G_\Q$ and $b_\sigma c_\tau\equiv 0\mod \mathfrak{a}$.
    \end{enumerate}
    Let $C=\{c_\sigma\mid \sigma\in \ir[G_\Q]\}$
    be the $\ir$-submodule in $K$
    generated by all $c_\sigma\in G_\Q$. 
    Then $C$ is nonzero because  $\rho$ is irreducible and
    it is actually a fractional ideal since the Galois group is compact.
    We put $\mathcal{L}_1=\ir v_1, \mathcal{L}_2=C v_2$ and $\mathcal{L}=\mathcal{L}_1\oplus\mathcal{L}_2$,
    which is the stable lattice generated by $v_1$ over $\ir[G_\Q]$.
    By above, the reduction of $\mathcal{L}$ modulo $\mathfrak{a}$ is an extension
        \begin{equation}\label{ext}
            0\to \ir/\mathfrak{a}(\chi_2)\cong \overline{\mathcal{L}}_2\to\overline{\mathcal{L}}
            \to \overline{\mathcal{L}}_1\cong \ir/\mathfrak{a}(\chi_1)\to 0
        \end{equation}
    as $C/\mathfrak{a}C\cong \ir/\mathfrak{a}$.
    We claim that $\overline{\mathcal{L}}$ has no quotient on which
    $G_\Q$ acts by $\chi_2$. Otherwise
    \begin{equation*}
        (\rho(\sigma_0)-\chi_2(\sigma_0)) v_1\equiv (\alpha_1-\alpha_2)v_1\mod \mathfrak{a},
    \end{equation*}
    and therefore $v_1$ lies in the kernel, which contradicts that $v_1$ generates $\mathcal{L}$.
    In particular, the extension gives a nontrivial class in
    $\Ext^1_{G_\Q}(\overline{\chi}_1,\overline{\chi}_2)=H^1(\Q, \ir/\mathfrak{a}(\chi_2\chi_1^{-1}))$.
\end{proof}
\begin{remark}
    The proposition was proved for the more general setting
    when $\ir$ is local Henselian and $\overline{\rho}^{ss}$
    is the sum of mutually non-isomorphic irreducible representations.
    We refer to \cite[chapter 1]{bell}
    for a formulation of these results in terms of 
    pseudo-representations and generalized matrix algebras.
\end{remark}

Apply the construction to $\mathfrak{a}=(\varpi)$ and $(\chi_1,\chi_2)=(1,\chi^{k-1})$,
we can obtain a nontrivial class in $H^1(\Q,\mathbb{F}(\omega^{k-1}))$ or $H^1(\Q,\mathbb{F}(\omega^{1-k}))$.
In either case, the restriction to $I_\ell$ for $\ell\neq p$ is
trivial since $\rho_f$ is unramified away from $p$ and
the image of $\mathcal{L}_1$ in $\overline{\mathcal{L}}$
is a section of the extension in (\ref{ext}).
To finish the proof, we need to use the fact that $a_p(f)\equiv 1+p^{k-1}\equiv 1$
is a unit and thus $f$ is ordinary.
\begin{theorem}
    We say an eigenform $f$ is $p$-ordinary if $a_p(f)$ is a $p$-unit.
    When this is the case,
    \begin{equation*}
        \rho_f\vert_{I_p}\sim \mat{\chi^{k-1}& *\\&1}.
    \end{equation*}
\end{theorem}
Now, if the reduction of the lattice $\mathcal{L}$ is
a non-split extension $0\to \mathbb{F} \to\overline{\mathcal{L}}
\to \mathbb{F}(\omega^{k-1})\to 0$
and $\mathcal{F}\subset K^2$ is the subspace where $I_p$ acts by $\chi^{k-1}$,
the reduction of $\mathcal{L}\cap F$ is a section of the extension
when restricted to $I_p$. Therefore we have show that
\begin{equation*}
    0\neq [\overline{\mathcal{L}}]\in H^1_f(\Q, \mathbb{F}(\omega^{1-k}))
    \cong \Hom(A_{p-k}, \mathbb{F}).
\end{equation*}
This completes the proof of Ribet's converse theorem.

\section{Congruence modules}
What happens when $\zeta(1-k)/2$ is divisible by higher powers of $p$?
It is natural to expect that we will be able to construct non-split extensions in larger coefficients.
Indeed, if the congruence relation (\ref{cong}) holds for $\mathfrak{a}=(\varpi^n)$,
the same arguments will give a nontrivial class in
\begin{equation*}
    H^1_f(\Q, \ir/\mathfrak{a}(\chi^{1-k}))\coloneqq
    \ker\bigg(
    H^1(\Q,\ir/\mathfrak{a}(\chi^{1-k}))\to\prod_{\ell}
    H^1(I_\ell,\ir/\mathfrak{a}(\chi^{1-k}))
    \bigg).
\end{equation*}
However, the Deligne-Serre lifting lemma only works for prime ideals,
and we have no control of the sizes of congruences for each minimal prime below $\mathfrak{m}=\ker(\eta)$.
We therefore need to consider all these primes, or equivalently
all the Hecke eigenforms congruent to $E_k$, at the same time.

Suppose $p^r\parallel\zeta(1-k)/2$. The ring homomorphism
\begin{equation*}
    \mathbb{T}\to \Z_p/(p^r),\quad T_\ell\mapsto 1+\ell^{k-1}\mod (p^r)
\end{equation*}
factors through the ideal $I\coloneqq(T_\ell-1-\ell^{k-1})\subset \mathbb{T}$. 
Let $J$ be the kernel of the surjective homomorphism
$\Z_p\to\mathbb{T}/I$, the following commutative diagram
shows that $J\subset (p^r)$.
\begin{equation*}
    \begin{tikzcd}
        & \mathbb{T} \arrow[r,"\eta"]\arrow[d]/I & \mathbb{T}/\mathfrak{m}\arrow[d,"\sim",sloped]\\
         \Z_p/J \arrow[r] \arrow[ur,"\sim",sloped] &\Z_p/(p^r) \arrow[r]& \mathbb{F}_p\\
    \end{tikzcd}
\end{equation*}
Let $\mathbb{T}_\mathfrak{m}$ be the localization, then the components of $K\coloneqq \mathbb{T}_\mathfrak{m}\otimes\Q_p=\prod_\lambda K_\lambda$
corresponds to conjugacy classes of eigenforms that are congruent to $E_k$.
If $\rho_\lambda$ is the p-adic Galois representation associated to each eigenform, we write
\begin{equation*}
    \rho_\mathfrak{m}=\prod \rho_\lambda \colon G_\Q\to \gl_2(K)=\prod \gl_2(K_\lambda),
\end{equation*}
which is irreducible at each component and 
$\tr\rho_\mathfrak{m}(\frob_\ell)=T_\ell\in \mathbb{T}_\mathfrak{m}\subset K$ if $\ell\neq p$. Since by tautology
\begin{equation*}
    \tr\rho_\mathfrak{m}(\frob_\ell)=T_\ell \equiv \ell^{k-1}+1=\chi^{k-1}(\frob_\ell)+1\mod I,
\end{equation*}
we have $\tr\rho_\mathfrak{m}(\sigma)\equiv \chi^{k-1}(\sigma)+1\mod I$
for $\sigma\in G_\Q$.
Apply the same argument as in the proof of the lattice construction, we can
find a basis $\{v_1,v_2\}$ of eigenvectors for some $\rho_\mathfrak{m}(\sigma_0)$ such that 
\begin{equation*}
    \rho_\mathfrak{m}(\sigma)=\mat{a_\sigma & b_\sigma\\c_\sigma & d_\sigma}\quad
    \text{satisfies}
\end{equation*}
\begin{enumerate}
    \item $a_\sigma\in \mathbb{T}_\mathfrak{m}$ for all $\sigma\in G_\Q$ and $a_\sigma\equiv \chi^{k-1}(\sigma)\mod I$,
    \item $d_\sigma \in \mathbb{T}_\mathfrak{m}$ for all $\sigma\in G_\Q$ and $d_\sigma\equiv 1\mod I$,
    \item $b_\sigma c_\tau\in \mathbb{T}_\mathfrak{m}$
    for all $\sigma, \tau\in G_\Q$ and $b_\sigma c_\tau\equiv 0\mod I$.
\end{enumerate}
Let $C=\{c_\sigma\mid \sigma\in \mathbb{T}_\mathfrak{m}[G_\Q]\}$
be the $\mathbb{T}_\mathfrak{m}$-submodule in $K$
generated by all $c_\sigma\in G_\Q$.
Since each $\rho_\lambda$ is an irreducible Galois representation,
the projection of $C$ to each $K_\lambda$ is a nonzero fractional ideal.
In particular, $C$ is a finite faithful $\mathbb{T}_\mathfrak{m}$-module.


Put $\mathcal{L}_1=\mathbb{T}_\mathfrak{m} v_1, \mathcal{L}_2=C v_2$ and $\mathcal{L}=\mathcal{L}_1\oplus\mathcal{L}_2$.
Again, $\mathcal{L}$ is the stable lattice generated by $v_1$ over $\mathbb{T}_\mathfrak{m}[G_\Q]$ and
the reduction modulo $I$ is an extension
\begin{equation*}
    0\to C/IC\cong \overline{\mathcal{L}}_2 \to\overline{\mathcal{L}}
    \to \overline{\mathcal{L}}_1\cong \mathbb{T}_\mathfrak{m}/I(\chi^{k-1})\to 0
\end{equation*}
having no quotient on which $G_\Q$ acts trivially. 
Since $\mathbb{T}_\mathfrak{m}/I=\Z_p/J$,
for any $\phi\in \Hom(C/IC, \Q_p/\Z_p)$
the non-split extension
\begin{equation*}
    0\to \overline{\mathcal{L}}_2/\ker(\phi) \to \overline{\mathcal{L}}/\ker(\phi)\to \overline{\mathcal{L}}_1\to 0
\end{equation*}
gives a nontrivial class in $H_f^1(\Q, \overline{\mathcal{L}}_2/\ker(\phi)(\chi^{1-k}))\hookrightarrow 
H_f^1(\Q, \Q_p/\Z_p(\chi^{1-k}))$.
Thus the map 
\begin{equation*}
    \Hom(C/IC, \Q_p/\Z_p)\hookrightarrow H^1_f(\Q, \Q_p/\Z_p(\chi^{1-k}))
\end{equation*}
is injective and dually we have a
surjective homomorphism
$H^1_f(\Q, \Q_p/\Z_p(\chi^{1-k}))^\vee\twoheadrightarrow C/IC$
between finitely-generated $Z_p$-modules.

\begin{definition}
    Let $M$ be an $R$-module of finite presentation
    \begin{equation*}
        R^a\xrightarrow{h} R^b \to M\to 0.
    \end{equation*}
    We define the ($0$-th) Fitting ideal $\fitt_R(M)$ to be 
    the $R$-ideal generated by the determinants of all $(b,b)$-minors in $h$
    if $a\geq b$, and $\fitt_R(M)=R$ if $a<b$. The definition
    is independent of the choice of the presentation.
\end{definition}
We also recall the following facts from \cite[Appendix]{Mazur1984}.
\begin{enumerate}
    \item $\fitt(M)\subset \fitt(M')$ if $M\twoheadrightarrow M$.
    \item $\fitt(M)\subset \text{Ann}(M)$, therefore the Fitting ideal of a faithful $R$-module is trivial.
    \item $\fitt_{R/I}(M/IM)=\fitt_R(M)\mod I$.
\end{enumerate}
We can deduce from which that $\fitt_{\Z_p} H^1_f(\Q,\Q_p/\Z_p(\chi^{1-k})^\vee \subset \fitt_{\Z_p}(C/IC)$
and
\begin{equation*}
    \fitt_{\Z_p}(C/IC) \mod J=\fitt_{\Z_p/J}(C/IC)=\fitt_{\mathbb{T}_\mathfrak{m}/I}(C/IC)=\fitt_{\mathbb{T}_\mathfrak{m}}(C)\mod I.
\end{equation*}
But $\fitt_{\mathbb{T}_\mathfrak{m}}(C)=0$ since $C$ is a faithful $\mathbb{T}_\mathfrak{m}$-module, thus
\begin{equation*}
    \fitt_{\Z_p}(C/IC)\subset J\subset (p^r)=(\zeta(1-k)).
\end{equation*}
And as $\#C/IC=\# \Z_p/\fitt_{\Z_p}(C/IC)\geq \#\Z_p/(\zeta(1-k)$,
we obtain the following proposition that partially answers
the question we posed in the beginning of the section.
\begin{proposition}
   Let $k\geq 4$ be an even number and $(p-1)\nmid k$, we have
   \begin{equation*}
       \#H^1_f(\Q,\Q_p/\Z_p(\chi^{1-k}))\geq \#\Z_p/(\zeta(1-k)).
   \end{equation*}
\end{proposition}


At last, we recall the definition of congruence modules and 
reinterpret the above chain of inclusions.
Let $M'=M'_k(1,\Z_p)$ be the space of modular forms
with $a_n(f)\in \Z_p$ for all $n\geq 1$ and 
$\mathbb{T}'$ be the Hecke algebra acting on $M'$.
Note that we do not require the constant term to be $p$-integral.
Since $S\subset M'$, the Hecke algebra $\mathbb{T}$ is a quotient of $\mathbb{T}'$.
In fact, we have
\begin{equation*}
    \mathbb{T}'\otimes \Q_p\cong \Q_p\times (\mathbb{T}\otimes \Q_p)
\end{equation*}
where the $\Q_p$-component is given by the Eisenstein series $E_k$.
Let $e\in \mathbb{T}'\otimes \Q_p$ be the idempotent corresponding to $(\mathbb{T}\otimes \Q_p)$, then $e\mathbb{T}'=\mathbb{T}$.
\begin{definition}
    Following \cite{Ti},
    we define the congruence modules
    \begin{equation*}
    C_0(\mathbb{T}')=e\mathbb{T}'/e\mathbb{T}'\cap \mathbb{T}',\quad
    C_0(M')=eM'/eM'\cap M'.
    \end{equation*}
\end{definition}
Since $e\mathbb{T}'\cap \mathbb{T}'$ is the kernel of the homomorphism to the Eisenstein component
\begin{equation*}
    \mathbb{T}'\to \Q_p,\quad T_\ell\mapsto 1+\ell^{k-1},
\end{equation*}
its image in $e\mathbb{T}'=\mathbb{T}$ is 
precisely $I=(T_\ell-1-\ell^{k-1})$ and we have recovered $C_0(\mathbb{T}')\cong \mathbb{T}/I$.
On the other hand, we have $eM'\cap M'=S$ and $C_0(M')$ measures the congruences between cusp forms and the Eisenstein series $E_k$.
\begin{lemma}
    The congruence module $C_0(M')$ has an element of order $p^n$
    if and only if
    there exists $G\in M'$ such that
    $F\coloneqq E_k-p^nG\in S$.
\end{lemma}
\begin{proof}
    Since $C_0(M')\cong M'/(\Z_pE_k\oplus S)$,
    if $G\in M'$ projects to an element of order $p^n$ in $C_0(M')$,
    then $p^nG=aE_k+F$ for some $a\in \Z_p^\times$ and $F\in S\setminus pS$.
    The converse is then also clear.
\end{proof}
In particular, recall that we have constructed
$F=E_k-\frac{\zeta(1-k)}{2}(240E_4)^a(-504E_6)^b\in S$.
By the lemma there exists a submodule
isomorphic to $\Z_p/(\zeta(1-k))$ in $C_0(M')$. We now observe that
\begin{align*}
    \text{Ann}_{\Z_p}C_0(M')\subset(\zeta(1-k))&,\quad \text{since }\Z_p/(\zeta(1-k))\hookrightarrow C_0(M'),\\
    J=\text{Ann}_{\Z_p}C_0(\mathbb{T}')\subset\text{Ann}_{\Z_p}C_0(M')&,\quad
    \text{since }C_0(\mathbb{T}')\otimes M'\twoheadrightarrow C_0(M'),
\end{align*}
and we have already proved $\fitt_{\Z_p} H^1_f(\Q, \Q_p/\Z_p(\chi^{1-k}))^\vee\subset J$.
In conclusion, we have
\begin{equation*}
    (\zeta(1-k))\supset \text{Ann}_{\Z_p}C_0(M')\supset \text{Ann}_{\Z_p}C_0(\mathbb{T}')\supset
    \fitt_{\Z_p} H^1_f(\Q, \Q_p/\Z_p(\chi^{1-k}))^\vee.
\end{equation*}
This will turn out to be a recurring theme.






\bibliographystyle{alpha}
\bibliography{biblio.bib}
\end{document}