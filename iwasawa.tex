\documentclass[leqno]{amsart}
\usepackage{amssymb}
\usepackage{amsmath} 
\usepackage{enumitem}
\usepackage{hyperref}
\usepackage{mathrsfs}
\usepackage{color}
\usepackage{mathtools,caption,bbm,euscript}
\usepackage[table,dvipsnames]{xcolor}
\usepackage{tikz-cd,tikz}
\usetikzlibrary{patterns}
\usepackage[utf8]{inputenc}
\usepackage[OT2,T1]{fontenc}
\hypersetup{
 colorlinks=true,
 linkcolor=DarkOrchid,
 filecolor=blue,
 citecolor=olive,
 urlcolor=orange,
 pdftitle={Notes on Iwasawa theory},
 %pdfpagemode=FullScreen,
 }
\usepackage{booktabs}

%[label=(\alph*)]
%[label=(\Alph*)]
%[label=(\roman*)]
%[label={(\bfseries R\arabic*)}]


\setlength{\textwidth}{\paperwidth}
\addtolength{\textwidth}{-2in}
\calclayout


\newcommand{\smat}[1]{\left( \begin{smallmatrix} #1 \end{smallmatrix} \right)}
\newcommand{\mat}[1]{\left( \begin{smallmatrix} #1 \end{smallmatrix} \right)}
\newcommand{\dBr}[1]{\llbracket{#1}\rrbracket}
\newcommand{\leg}[2]{\left(\frac{#1}{#2}\right)}

% double bracket
\makeatletter
\newsavebox{\@brx}
\newcommand{\llangle}[1][]{\savebox{\@brx}{\(\m@th{#1\langle}\)}%
  \mathopen{\copy\@brx\kern-0.5\wd\@brx\usebox{\@brx}}}
\newcommand{\rrangle}[1][]{\savebox{\@brx}{\(\m@th{#1\rangle}\)}%
  \mathclose{\copy\@brx\kern-0.5\wd\@brx\usebox{\@brx}}}
  \newcommand{\llbracket}[1][]{\savebox{\@brx}{\(\m@th{#1[}\)}%
  \mathopen{\copy\@brx\kern-0.5\wd\@brx\usebox{\@brx}}}
\newcommand{\rrbracket}[1][]{\savebox{\@brx}{\(\m@th{#1]}\)}%
  \mathclose{\copy\@brx\kern-0.5\wd\@brx\usebox{\@brx}}}
\makeatother

%%% Unitary group specific
\newcommand{\V}{\mathbf V} 
\newcommand{\W}{\mathbf W} 
\newcommand{\G}{\mathbf G} %GU(2,2)
\newcommand{\X}{\mathbf H} %Hermitian symmetric domain
\newcommand{\KK}{\mathbf K} %compact open subgroup
\newcommand{\xx}{\mathbf x}
\newcommand{\yy}{\mathbf y}
\newcommand{\nn}{\mathbf n} %unipotent

\newcommand{\qdr}[1]{\underline{ #1 }}
\newcommand{\cA}{\mathcal A} %complex abelian varieties
\newcommand{\cB}{\mathcal B} %complex abelian varieties
\newcommand{\bB}{\mathbf B}

\newcommand{\mm}{\mathbf{m}}


\newcommand{\balpha}{\boldsymbol{\alpha}} %diagonal torus elements
\newcommand{\bbeta}{\boldsymbol{\beta}} %diagonal torus elements
\newcommand{\bnu}{\boldsymbol{\nu}} %character on diagonal torus
\newcommand{\bmu}{\boldsymbol{\mu}} %character on diagonal torus

%%% modular form specific

\newcommand{\wt}[1]{\underline{ #1 }}
\newcommand{\bwt}[1]{\underline{\boldsymbol { #1 }}}
\newcommand{\M}{\mathbf{M}}
\newcommand{\Mk}{\mathbf{M}_{\wt{k}}}
\newcommand{\pM}{\widehat{\mathbf{M}}}
\newcommand{\pMk}{\widehat{\mathbf{M}}_{\wt{k}}}
\newcommand{\AM}{\mathcal{A}}
\newcommand{\bun}{\mathcal{E}}
\newcommand{\ii}{\mathbf i}
\newcommand{\ff}{\mathbf f}
\newcommand{\her}{\EuScript{H} }



\newcommand{\phk}{\mathcal{H}_{\wt{k}}} %Hecke algebra


\newcommand{\vk}{v_{-\wt{k}}} % lowest weight vector
\newcommand{\lk}{\mathnormal{l}_{\wt{k}}} % projection to lowest weight
\newcommand{\blk}{\mathnormal{l}_{\bwt{k}}} % projection to lowest weight
\newcommand{\Lk}{L_{\wt{k}}} 
\newcommand{\pf}{\hat{f}} % algebraic modular form
\newcommand{\pfk}{\hat{f}_{\wt{k}}}
\newcommand{\euF}{\EuScript{F}} % Hida family
\newcommand{\B}{\mathbf{B}} % pairing of algebraic modular form


\newcommand{\ww}{\boldsymbol{\omega}}
\DeclareMathOperator{\an}{an}
\DeclareMathOperator{\ordp}{{ord}_p}
\DeclareMathOperator{\cts}{cts}
\DeclareMathOperator{\Frob}{Frob}


%%% Linear algebraic groups
\DeclareMathOperator{\GL}{GL}
\DeclareMathOperator{\SL}{SL}
\DeclareMathOperator{\Sp}{Sp}
\DeclareMathOperator{\Mp}{Mp}
\DeclareMathOperator{\GSp}{GSp}
\DeclareMathOperator{\UU}{U}
\DeclareMathOperator{\GUU}{GU}
\DeclareMathOperator{\gl}{\mathfrak{gl}}
\DeclareMathOperator{\mtr}{tr}
\DeclareMathOperator{\diag}{diag}
\DeclareMathOperator{\Ad}{Ad}
\DeclareMathOperator{\vol}{vol}

\DeclareMathOperator{\val}{val}
\DeclareMathOperator{\odp}{ord}
\DeclareMathOperator{\Lie}{Lie}
\DeclareMathOperator{\Pol}{Pol_p}

%%% Adelic rings
\newcommand{\Q}{{\mathbf{Q}}}
\newcommand{\Z}{{\mathbf{Z}}}
\newcommand{\Qp}{\mathbf{Q}_p}
\newcommand{\Zp}{\mathbf{Z}_p}
\newcommand{\Ql}{\mathbf{Q}_\ell}
\newcommand{\Zl}{\mathbf{Z}_\ell}
\newcommand{\R}{\mathbf R}
\newcommand{\C}{\mathbf C}
\newcommand{\A}{\mathbf A}
\newcommand{\hZ}{{\hat{\mathbf{Z}}}}
\newcommand{\dd}{\mathfrak{d}} %different
\newcommand{\DD}{\mathcal{D}}  %discriminant

\newcommand{\arch}{\mathbf{a}}
\newcommand{\fin}{\mathbf{h}}

\newcommand{\F}{{\mathcal{F}}} %global 
\newcommand{\OF}{{\mathcal{O}_{\F}}}
\newcommand{\K}{{\mathcal{K}}} %global quadratic
\newcommand{\OK}{\mathcal{O}_{\K}}
\newcommand{\kk}{F} %local
\newcommand{\E}{E} %local quadratic


\DeclareMathOperator{\Sel}{Sel}
\DeclareMathOperator{\Gal}{Gal}
\DeclareMathOperator{\Nr}{\mathsf{N}}
\DeclareMathOperator{\Tr}{Tr}
\newcommand{\qch}{\epsilon} % quadratic character of K/F


%%% Fonts
\newcommand{\oeu}{\EuScript{O}}
\newcommand{\eeu}{\EuScript{E}}
\newcommand{\feu}{\EuScript{F}}
\newcommand{\geu}{\EuScript{G}}
\newcommand{\keu}{\EuScript{K}}

\newcommand{\oo}{\mathcal O}
\newcommand{\bs}{\mathcal S}
\newcommand{\id}{\mathbf{1}}

\newcommand{\1}{\mathbf{1}} 
\newcommand{\bfe}{\mathbf e}
\newcommand{\bff}{\mathbf f}

\newcommand{\bX}{\mathbb{X}}
\newcommand{\bY}{\mathbb{Y}}
\newcommand{\bV}{\mathbb{V}}
\newcommand{\bW}{\mathbb{W}}

\newcommand{\fa}{\mathfrak a}
\newcommand{\fg}{\mathfrak g}
\newcommand{\fc}{\mathfrak c}
\newcommand{\fs}{\mathfrak s}
\newcommand{\fm}{\mathfrak m}
\newcommand{\fn}{\mathfrak n}
\newcommand{\fl}{\mathfrak l}
\newcommand{\fp}{\mathfrak p}
\newcommand{\bfp}{\overline{\mathfrak p}}
\newcommand{\fq}{\mathfrak q}
\newcommand{\bfq}{\overline{\mathfrak q}}

\newcommand{\btheta}{\boldsymbol{\theta}}
\newcommand{\bdelta}{\boldsymbol{\delta}}


\newcommand{\fG}{\mathfrak{G}}
\newcommand{\fX}{\mathfrak{X}}
\newcommand{\euW}{\EuScript{W}}


%%% Categorical
\DeclareMathOperator{\Ext}{Ext}
\DeclareMathOperator{\End}{End}
\DeclareMathOperator{\Hom}{Hom}
\DeclareMathOperator{\Inj}{Inj}
\DeclareMathOperator{\Isom}{Isom}
\DeclareMathOperator{\Aut}{Aut}
\DeclareMathOperator{\Ind}{Ind}
\DeclareMathOperator{\coker}{coker}
\DeclareMathOperator{\rank}{rank}
\DeclareMathOperator{\corank}{corank}


\DeclareMathOperator{\Res}{Res}
\DeclareMathOperator{\rec}{rec}



\newcommand{\bw}{{w^c}}
\newcommand{\ee}{\mathbf e}


\newtheorem*{theorem*}{Theorem}
\newtheorem{thm}{Theorem}[section]
\newtheorem{lem}[thm]{Lemma}
\newtheorem{prop}[thm]{Proposition}
\newtheorem{cor}[thm]{Corollary}


\theoremstyle{definition}
\newtheorem{definition}[thm]{Definition}
\newtheorem{defn}[thm]{Definition}
\theoremstyle{remark}
\newtheorem{rem}[thm]{Remark}
\newtheorem*{Remark*}{Remark}
\newtheorem{ack}{Acknowledgement}

\newcommand{\red}[1]{\textcolor{Red}{#1}}



\begin{document}
\title{Notes on Iwasawa theory}
\author[Y-S.~Lee]{Yu-Sheng Lee}
\address{Department of Mathematics, University  of Michigan, Ann Arbor, MI 48109, USA}
\email{yushglee@umich.edu}
\date{\today}

\maketitle
\setcounter{tocdepth}{1}
\tableofcontents


\section{Elliptic curves with CM}

\[
\begin{tikzpicture}
	\draw(0,3)--(0,-2) (-4,0)--(3,0);
	\draw[dashed](-2,-2)--(3,3) (1,-2)--(-3,2);
	\path[pattern=north east lines] 
		(-3,0)--(-3,2)--(-1,2)--(-1,0)--cycle;
	\path[pattern=north east lines] 
		(0,-1)--(0,-2)--(2,-2)--(2,-1)--cycle;
\end{tikzpicture}
\]



\section{$L$-invariant}
Let $E/\Q$ be an elliptic curve, assume that
\begin{enumerate}[label=(\alph*)]
	\item $E$ has stable reduction modulo $p$
	\item  $E$ has split multiplicative reduction at  $p$.
\end{enumerate}
Tate's $p$-adic uniformization thoery, $q_E\in p\Zp$
such that  $E(\bar{\Q}_p)\cong \bar{\Q}_p^\times/q_E^{\Z}$, 
which is defined over $\Qp$. 
\[
	 \mathcal{L}_p(E)\coloneqq \frac{\log_p(q_E)}{\ordp(q_E)}
\]

\begin{thm}
	If $p\geq 5$, then
	 \[
		 L'_p(E,1)=\mathcal{L}_p(E)\cdot 
		 \frac{L_\infty(E,1)}{\Omega_\infty}
	\]
\end{thm}

Under the assumption that $E$ has split multiplicative reduction
\[
	L_p(E,2-s)=w_p \langle N\rangle^{s-1}L_p(E,s)
\]
where $w_p=-w_\infty$.
\begin{itemize}
	\item if $w_\infty=-1$, then trivial since
		$L'_p(E,1)=0$ then.
	\item if  $w_\infty=1$,  
		then $L_p(E,s)$ has odd order at $s=1$.
		The order is one iff
		$L_\infty(E,1)\neq 0$ and  $\log_p(q_E)\neq 0$.
\end{itemize}
Conjecturally
\[
	ord_{s=1}(L_p(E,s))=
	ord_{s=1}(L_\infty(E,s))=+1
\]

As an example, let $E=X_0(11)$ and  $p=11$.
Let 
 \[
	 \rho_E\colon G_\Q\to \Aut(T_pE),\quad
	 f_E=q\prod(1-q^n)^2(1-q^{11n})^2 
	 \text{ is ordinary at }p.
\]
The universal ordinary Hecke algebra is 
$\Lambda=\Zp\llbracket 1+p\Zp\rrbracket$.
By Hida's theory, there exists a free rank two $\Lambda$-module
$\mathbf{T}$ and 
\[
	\rho\colon G_\Q\to \Aut_\Lambda(\mathbf{T}),\quad
	\mathbf{T}/P_0\mathbf{T}\cong T_pE
\]
where $P_0\subset\Lambda$ is the augmentation ideal.
 \begin{enumerate}[label=(\alph*)]
	\item for $k\in \Zp$,
		let $\sigma_{k-2}\colon\Lambda\to\Zp$
		extends $t\mapsto t^{k-2}$ on $1+p\Zp$.
		Let  $P_k=\ker(\sigma_{k-2})$
	\item $\rho_2=\rho_E$
	\item for each integer  $k\geq 2$,
		there is a new form  $f_k$
		of conductor dividing  $p$
		whose Galois representation is 
		$\mathbf{T}_k=\mathbf{T}/P_k\mathbf{T}$
		 \[
			\text{conductor}=1 \Longleftrightarrow
			k>2 \text{ and } k\equiv 2\mod p-1
		\]
		Otherwise, the condcutor is $p$
		and the Nebentypus is  $\omega^{2-k}$.
	\item 
		\[
			\rho\vert_{G_{\Qp}}\sim 
			\smat
			{\chi\varphi^{-1} & * \\
			0 & \varphi }
		\]
		where $\varphi\colon G_{\Qp}\to \Lambda^\times$
		is unramified and $\chi\colon G_{\Qp}\to \Lambda^\times$
		is such that 
		$\sigma_{k-2}\circ \chi=\chi_0^{k-2}\omega^{2-k}$ 
		where $\chi_0$ is the cyclotomic character.
\end{enumerate}
Let $a_p=\varphi(\Frob_p)\in \Lambda^\times$
The $p$-factor of  $L_\infty(f_k,s)$ is
 \[
	 [(1-\alpha_kp^{-s})(1-\beta_kp^{-s})]^{-1},\quad
	 \alpha_k=a_p(k)\coloneqq \sigma_{k-2}(a_p),
	 \beta_k=
	 \begin{cases}
		 p^{k-1}/\alpha_k, & k>2\text{ and }k\equiv 2\mod p-1,\\
		 0, & \text{otherwise}.
	 \end{cases}
\]
Then  $\alpha_k\equiv \alpha_2=1$ (interpolated by $a_p$ )
and  $\beta_k\equiv 0\mod p$ (cannot be interpolated).
\[
	\mathbf{f}\colon \sum a_nq^n\in \Lambda\llbracket q\rrbracket
	\Longrightarrow
	\mathbf{f}_k=\sum a_n(k)q^n= f_k^*\coloneqq
	f_k(z)-\beta_kf_k(pz)
\]

There is a two variable $p$-adic L-function $L_p(k,s), k,s\in \Zp$.
 \begin{enumerate}[label=(\alph*)]
	 \item $L_p(k,s)$ is analytic for  $k,s\in \Zp$
	 \item  $L_p(2,s)=L_p(E,s)$ and 
		 $L_p(k,s)$ is  $L_p(f_k,s)$.
		 \[
			 L_p(k,s)=\Omega_k\cdot L_p(f_k,s),\Omega_k\in \Qp
			 \text{ and } \Omega_2=1
		 \]
	 \item  $L_p(k,k-s)=-L_p(k,s)$
	 \item  $L_p(k,1)=(1-q_p(k)^{-1})L_p^*(k,1)$
		 where  $L_p^*(k,1)$ is a  $p$-adic analytic function 
		 such that 
		 \[
			 L^*_p(2,1)=\frac{L_\infty(E,1)}{\Omega_E}
		 \]
\end{enumerate}
 For $k\geq2, 0<s_0<k, s\equiv 1\mod p-1$,
  \[
	  L_p(f_k,s_0)=
	  (1-\beta_kp^{-s_0})(1-\alpha_k^{-1}p^{s_0-1})
	  \cdot \frac{L_\infty(f_k,s_0)}{\Omega_{f_k}}=
	  \frac{(1-\alpha_k^{-1}p^{s_0-1})}{(1-\alpha_kp^{-s_0})}
	  \cdot \frac{L^{(p)}_\infty(f_k,s_0)}{\Omega_{f_k}}=
 \]
When $s_0=1$, the factor 
$1-a_p(k)^{-1}$ can be interpolated and is a non-unit,
so $L_p(k,1)$ is divisible by which, the quotient 
$L_p^*(k,1)$ is 
\[
	  L_p(f_k,s_0)=
	  (1-\beta_kp^{-s_0})
	  \cdot \frac{L_\infty(f_k,s_0)}{\Omega_{f_k}}=
\]
When $k-2$, reduces to
$L_p^*(2,1)=\frac{L_\infty(E,1)}{\Omega_E}$ since 
$\beta_2=0,\Omega_{f_2}=\Omega_E, \Omega_2=1$.

By functional equation $L_p(k,k/2)=0$,
at  $(k,s)=(2,1)$,
 \[
	 L_p(k,s)\sim c\cdot (-\frac{1}{2}(k-2)+(s-1))\quad c\in \Zp
\]
Comparing and get 
\[
	L'_p(E,1)=-2a'_p(2)\cdot 
	\frac{L_\infty(E,1)}{\Omega_E}.
\]
It remains to prove $\mathcal{L}_p(E)=-2a'_p(2)$,
recall that  $a_p(2)=0$.

From the uniformization, let $V=V_p(E)$
 \[
	 0\to \Qp(1)\to V\to \Qp\to 0
\]
determines a class $\xi\in H^1(G_{\Qp},\Qp(1))$.
\[
	(\ordp,\log_p)\colon H^1(\Qp(1))\cong \Qp^2
\]
so $\mathcal{L}_p(E)$ is the slope of the line  $\xi$.

 \[
	 0\to \Lambda(\chi\psi)\to \mathbf{T}(\varphi^{-1})\to \Lambda\to 0
	 \Longrightarrow
	 0\to \Qp(\chi_k\psi_k)\to \mathbf{V}_k\to \Qp\to 0
	 \Longrightarrow
	 \xi_k\in H^1(\Qp(\chi_k\psi_k))
\]
where $\psi=\varphi^{-2}$.
Clearly(?) $\xi_k\neq 0$ if  $k$ is close to  $2$,
but $H^1(\Qp(\chi_k\psi_k)$ is one-dimensional  when $k\neq 2$ 
\[
	\frac{d\psi_k(\Frob_p)}{dk}\vert_{k=2}=\mathcal{L}_p(E)
\]

In general, 
start with newform of weight $2$ over  $\Gamma_1(Np)(p\nmid N)$
with split multiplicative reduction at $p(a_p(f)=1)$. 


%\bibliographystyle{amsalpha}
%\bibliography{biblio}
\end{document}
