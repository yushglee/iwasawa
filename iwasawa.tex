\documentclass[leqno]{amsart}
\usepackage{amssymb}
\usepackage{amsmath} 
\usepackage{enumitem}
\usepackage{hyperref}
\usepackage{mathrsfs}
\usepackage{color}
\usepackage{mathtools,caption,bbm,euscript}
\usepackage[table,dvipsnames]{xcolor}
\usepackage{tikz-cd,tikz}
\usetikzlibrary{patterns}
\usepackage[utf8]{inputenc}
\usepackage[OT2,T1]{fontenc}
\hypersetup{
 colorlinks=true,
 linkcolor=DarkOrchid,
 filecolor=blue,
 citecolor=olive,
 urlcolor=orange,
 pdftitle={Notes on Iwasawa theory},
 %pdfpagemode=FullScreen,
 }
\usepackage{booktabs}

%[label=(\alph*)]
%[label=(\Alph*)]
%[label=(\roman*)]
%[label={(\bfseries R\arabic*)}]

\DeclareSymbolFont{cyrletters}{OT2}{wncyr}{m}{n}
\DeclareMathSymbol{\Sha}{\mathalpha}{cyrletters}{"58}

\setlength{\textwidth}{\paperwidth}
\addtolength{\textwidth}{-2in}
\calclayout


\newcommand{\smat}[1]{\left( \begin{smallmatrix} #1 \end{smallmatrix} \right)}
\newcommand{\mat}[1]{\left( \begin{smallmatrix} #1 \end{smallmatrix} \right)}
\newcommand{\dBr}[1]{\llbracket{#1}\rrbracket}
\newcommand{\leg}[2]{\left(\frac{#1}{#2}\right)}

% double bracket
\makeatletter
\newsavebox{\@brx}
\newcommand{\llangle}[1][]{\savebox{\@brx}{\(\m@th{#1\langle}\)}%
  \mathopen{\copy\@brx\kern-0.5\wd\@brx\usebox{\@brx}}}
\newcommand{\rrangle}[1][]{\savebox{\@brx}{\(\m@th{#1\rangle}\)}%
  \mathclose{\copy\@brx\kern-0.5\wd\@brx\usebox{\@brx}}}
  \newcommand{\llbracket}[1][]{\savebox{\@brx}{\(\m@th{#1[}\)}%
  \mathopen{\copy\@brx\kern-0.5\wd\@brx\usebox{\@brx}}}
\newcommand{\rrbracket}[1][]{\savebox{\@brx}{\(\m@th{#1]}\)}%
  \mathclose{\copy\@brx\kern-0.5\wd\@brx\usebox{\@brx}}}
\makeatother

%%% Unitary group specific
\newcommand{\V}{\mathbf V} 
\newcommand{\W}{\mathbf W} 
\newcommand{\G}{\mathbf G} %GU(2,2)
\newcommand{\X}{\mathbf H} %Hermitian symmetric domain
\newcommand{\KK}{\mathbf K} %compact open subgroup
\newcommand{\xx}{\mathbf x}
\newcommand{\yy}{\mathbf y}
\newcommand{\nn}{\mathbf n} %unipotent

\newcommand{\qdr}[1]{\underline{ #1 }}
\newcommand{\cA}{\mathcal A} %complex abelian varieties
\newcommand{\cB}{\mathcal B} %complex abelian varieties
\newcommand{\bB}{\mathbf B}

\newcommand{\mm}{\mathbf{m}}


\newcommand{\balpha}{\boldsymbol{\alpha}} %diagonal torus elements
\newcommand{\bbeta}{\boldsymbol{\beta}} %diagonal torus elements
\newcommand{\bnu}{\boldsymbol{\nu}} %character on diagonal torus
\newcommand{\bmu}{\boldsymbol{\mu}} %character on diagonal torus

%%% modular form specific

\newcommand{\wt}[1]{\underline{ #1 }}
\newcommand{\bwt}[1]{\underline{\boldsymbol { #1 }}}
\newcommand{\M}{\mathbf{M}}
\newcommand{\Mk}{\mathbf{M}_{\wt{k}}}
\newcommand{\pM}{\widehat{\mathbf{M}}}
\newcommand{\pMk}{\widehat{\mathbf{M}}_{\wt{k}}}
\newcommand{\AM}{\mathcal{A}}
\newcommand{\bun}{\mathcal{E}}
\newcommand{\ii}{\mathbf i}
\newcommand{\ff}{\mathbf f}
\newcommand{\her}{\EuScript{H} }



\newcommand{\phk}{\mathcal{H}_{\wt{k}}} %Hecke algebra


\newcommand{\vk}{v_{-\wt{k}}} % lowest weight vector
\newcommand{\lk}{\mathnormal{l}_{\wt{k}}} % projection to lowest weight
\newcommand{\blk}{\mathnormal{l}_{\bwt{k}}} % projection to lowest weight
\newcommand{\Lk}{L_{\wt{k}}} 
\newcommand{\pf}{\hat{f}} % algebraic modular form
\newcommand{\pfk}{\hat{f}_{\wt{k}}}
\newcommand{\euF}{\EuScript{F}} % Hida family
\newcommand{\B}{\mathbf{B}} % pairing of algebraic modular form


\newcommand{\ww}{\boldsymbol{\omega}}
\DeclareMathOperator{\an}{an}
\DeclareMathOperator{\ordp}{{ord}_p}
\DeclareMathOperator{\cts}{cts}
\DeclareMathOperator{\Frob}{Frob}


%%% Linear algebraic groups
\DeclareMathOperator{\GL}{GL}
\DeclareMathOperator{\SL}{SL}
\DeclareMathOperator{\Sp}{Sp}
\DeclareMathOperator{\Mp}{Mp}
\DeclareMathOperator{\GSp}{GSp}
\DeclareMathOperator{\UU}{U}
\DeclareMathOperator{\GUU}{GU}
\DeclareMathOperator{\gl}{\mathfrak{gl}}
\DeclareMathOperator{\mtr}{tr}
\DeclareMathOperator{\diag}{diag}
\DeclareMathOperator{\Ad}{Ad}
\DeclareMathOperator{\vol}{vol}

\DeclareMathOperator{\val}{val}
\DeclareMathOperator{\odp}{ord}
\DeclareMathOperator{\Lie}{Lie}
\DeclareMathOperator{\Pol}{Pol_p}

%%% Adelic rings
\newcommand{\Q}{{\mathbf{Q}}}
\newcommand{\Z}{{\mathbf{Z}}}
\newcommand{\Qp}{\mathbf{Q}_p}
\newcommand{\Zp}{\mathbf{Z}_p}
\newcommand{\Ql}{\mathbf{Q}_\ell}
\newcommand{\Zl}{\mathbf{Z}_\ell}
\newcommand{\R}{\mathbf R}
\newcommand{\C}{\mathbf C}
\newcommand{\A}{\mathbf A}
\newcommand{\hZ}{{\hat{\mathbf{Z}}}}
\newcommand{\dd}{\mathfrak{d}} %different
\newcommand{\DD}{\mathcal{D}}  %discriminant

\newcommand{\arch}{\mathbf{a}}
\newcommand{\fin}{\mathbf{h}}

\newcommand{\F}{{\mathcal{F}}} %global 
\newcommand{\OF}{{\mathcal{O}_{\F}}}
\newcommand{\K}{{\mathcal{K}}} %global quadratic
\newcommand{\OK}{\mathcal{O}_{\K}}
\newcommand{\kk}{F} %local
\newcommand{\E}{E} %local quadratic


\DeclareMathOperator{\Sel}{Sel}
\DeclareMathOperator{\Gal}{Gal}
\DeclareMathOperator{\Nr}{\mathsf{N}}
\DeclareMathOperator{\Tr}{Tr}
\newcommand{\qch}{\epsilon} % quadratic character of K/F


%%% Fonts
\newcommand{\oeu}{\EuScript{O}}
\newcommand{\eeu}{\EuScript{E}}
\newcommand{\feu}{\EuScript{F}}
\newcommand{\geu}{\EuScript{G}}
\newcommand{\keu}{\EuScript{K}}

\newcommand{\oo}{\mathcal O}
\newcommand{\bs}{\mathcal S}
\newcommand{\id}{\mathbf{1}}

\newcommand{\1}{\mathbf{1}} 
\newcommand{\bfe}{\mathbf e}
\newcommand{\bff}{\mathbf f}

\newcommand{\bX}{\mathbb{X}}
\newcommand{\bY}{\mathbb{Y}}
\newcommand{\bV}{\mathbb{V}}
\newcommand{\bW}{\mathbb{W}}

\newcommand{\fa}{\mathfrak a}
\newcommand{\fg}{\mathfrak g}
\newcommand{\fc}{\mathfrak c}
\newcommand{\fs}{\mathfrak s}
\newcommand{\fm}{\mathfrak m}
\newcommand{\fn}{\mathfrak n}
\newcommand{\fl}{\mathfrak l}
\newcommand{\fp}{\mathfrak p}
\newcommand{\bfp}{\overline{\mathfrak p}}
\newcommand{\fq}{\mathfrak q}
\newcommand{\bfq}{\overline{\mathfrak q}}

\newcommand{\btheta}{\boldsymbol{\theta}}
\newcommand{\bdelta}{\boldsymbol{\delta}}


\newcommand{\fG}{\mathfrak{G}}
\newcommand{\fX}{\mathfrak{X}}
\newcommand{\euW}{\EuScript{W}}


%%% Categorical
\DeclareMathOperator{\Ext}{Ext}
\DeclareMathOperator{\End}{End}
\DeclareMathOperator{\Hom}{Hom}
\DeclareMathOperator{\Inj}{Inj}
\DeclareMathOperator{\Isom}{Isom}
\DeclareMathOperator{\Aut}{Aut}
\DeclareMathOperator{\Ind}{Ind}
\DeclareMathOperator{\coker}{coker}
\DeclareMathOperator{\rank}{rank}
\DeclareMathOperator{\corank}{corank}


\DeclareMathOperator{\Res}{Res}
\DeclareMathOperator{\res}{res}
\DeclareMathOperator{\rec}{rec}
\DeclareMathOperator{\rrho}{\bar{\rho}}



\newcommand{\bw}{{w^c}}
\newcommand{\ee}{\mathbf e}


\newtheorem*{theorem*}{Theorem}
\newtheorem{thm}{Theorem}[section]
\newtheorem{lem}[thm]{Lemma}
\newtheorem{prop}[thm]{Proposition}
\newtheorem{cor}[thm]{Corollary}


\theoremstyle{definition}
\newtheorem{definition}[thm]{Definition}
\newtheorem{defn}[thm]{Definition}
\theoremstyle{remark}
\newtheorem{rem}[thm]{Remark}
\newtheorem*{Remark*}{Remark}
\newtheorem{ack}{Acknowledgement}

\newcommand{\red}[1]{\textcolor{Red}{#1}}



\begin{document}
\title{Notes on Iwasawa theory}
\author[Y-S.~Lee]{Yu-Sheng Lee}
\address{Department of Mathematics, University  of Michigan, Ann Arbor, MI 48109, USA}
\email{yushglee@umich.edu}
\date{\today}

\maketitle
\setcounter{tocdepth}{1}
\tableofcontents


\section{Galois deformation}



Let $\Gamma$ be a group 
and  $\rrho\colon \Gamma\to \GL_k(V)$
be a finite-dimensional representation
over a finite field $k$.
\begin{lem}[Burnside theorem]
Let $R\subset \End_k(V)$
be the subalgebra generated by  $\rrho(\Gamma)$ over  $k$.
Then $R=\End_k(V)$ when 
$\rrho$ is absolutely irreducible.
\end{lem}
\begin{proof}
	Since $R\otimes_k\bar{k}\subset 
	\End_{\bar{k}}(V\otimes_k\bar{k})$,
	by dimension counting it suffices 
	to assume $k=\bar{k}$.
	Thus $k=\End(\rrho)=\End_R(V)$ since
	it is a division algebra
	over an algebraically closed field.
	And therefore $R=\End_k(V)$
	by Jacobson's density theorem.
\end{proof}
\begin{lem}[Schur's lemma]
	If $\rrho$ is absolutely irreducible,
	then $\End(\rrho)=k$.
\end{lem}
\begin{proof}
	This follows from Burnside's theorem
	since $\End(\rrho)$ is the center of $R=\End_k(V)$.
\end{proof}
\begin{lem}[Braur-Nesbitt]
	Let $\rrho'\colon \Gamma\to \GL_k(V')$
	be another representation.
	Suppose $n=\dim_kV=\dim_kV'<\textnormal{char}(k)$
	and  $\mtr\rrho=\mtr\rrho'$,
	then  $\rrho\cong\rrho'$
	if $\rrho$ is irreducible.
\end{lem}
\begin{proof}
	Let $\{V=V_1,\cdots,V_r\}$ 
	be the irreducible components 
	appearing in the semi-simplification
	of  $V\oplus V'$
	By Jacobson's density theorem,
	there exists projectors
	$e_1,\cdots,e_r\in k[\Gamma]$
	such that $e_i$ acts as the identity map on  $V_i$
	and trivially on  $V_j$ for  $i\neq j$.
	The lemma now follows from comparing
	$\mtr\rrho(e_i)=\mtr\rrho'(e_i)$.
\end{proof}
\begin{lem}[Carayol lemms]
	If $\rrho$ is absolutely irreducible
	and the image of $\mtr\rrho$
	lies in a subfield  $k'\subset k$,
	then $\rrho$ is isomorphic 
	to the base change of 
	a representation over $k'$.
\end{lem}
\begin{proof}
	With notations as above,
	let $\{y_1,\cdots,y_{n^2}\}\subset \rrho(\Gamma),
	n=\dim_kV$
	be a basis of $R=\End_k(V)$ over $k$.
	Write $y=\sum_{i=1}^{n^2}a_iy_i,a_i\in k$
	for any $y\in \rrho(\Gamma)$.
	Then $a_i\in k'$ for all  $i$
	since  $\mtr(yy_i)\in k'$
	and  $(\mtr(y_iy_j))\in \GL_{n^2}(k')$.
	Thus $\rrho(\Gamma)$ factors through
	the  $k'$-algebra $R'$ generated by  
	$\{y_1,\cdots,y_{n^2}\}$.


	Since $R'\otimes_{k'}k=R=\End_k(V)$
	is a central simple algebra over  $k$,
	$R'$ is a central simple algebra over  $k'$.
	Therefore $R'\cong M_n(k')$
	by Wedderburn-Artin and dimension counting.
	Let $\rrho'$ be the induced canonical 
	representation over  $k'$,
	then  $\rrho'\otimes_{k'}k\cong \rrho$
	by Skolem-Noether theorem.
\end{proof}


Now fix an absolutely irreducible $\rrho\colon \Gamma\to \GL_n(k)$.
The above results can be extends to 
deformations of  $\rrho$ to an
Artinian local ring $A$ with residue field  $k$.
\begin{lem}
	Let $\rho\colon \Gamma\to \GL_n(A)$
	be a deformation to $A$,
	and $R\subset M_n(A)$ be the $A$-subalgebra
	generated by $\rho(\Gamma)$ over $A$,
	then  $R=M_n(A)$.
\end{lem}
\begin{proof}
	This follows from Nakayama's lemma
	since $R/\fm R=M_n(k)$.
\end{proof}
\begin{lem}
	Let $\rho\colon \Gamma\to \GL_n(A)$	
	be a deformation to $A$,
	then $\End(\rho)=A$.
\end{lem}
\begin{proof}
	With notations as above,
	this follows from $R=M_n(A)$.
\end{proof}
\begin{lem}
	Suppse $\rho,\rho'\colon \Gamma\to \GL_n(A)$
	are deformations of  $\rrho$ to  $A$
	and  $\mtr\rho=\mtr\rho'$,
	then  $\rho\cong \rho'$.
\end{lem}
\begin{proof}
	This is proved by induction on $\ell(A)$.
	Let  $I\subset A$ be a minimal ideal
	so that $I\cong k$ as  $A$-modules.
	By induction
	$\rho(g)=\rho'(g)+\delta(g)$
	for  $\delta\colon \Gamma\to M_n(I)=M_n(k)$
	with $\mtr\delta=0$,
	from which 
	\[
		 \rrho(g_1)\delta(g_2)=\delta(g_1)\rrho(g_2)
		 \Longrightarrow
		 \mtr(\rrho(g_1)\delta(g_2))=0
		 \text{ when }
		 g_2\in \ker(\rrho).
	\]
	Since $\rrho(g_1)$ generates  $M_n(k)$,
	the above implies that  $\delta(g_2)=0$
	when  $g_2\in\ker(\rrho)$.
	Thus $\delta$ factors through
	a derivation on $M_n(k)$.
	Since any derivation is inner, 
	there exists  $U\in M_n(k)$
	such that  $\delta(g)=\rrho(g)U-U\rrho(g)$.
	Therefore $\rho'=(1-U)\rho(1+U)$.
\end{proof}
\begin{lem}
	Let $\rho\colon \Gamma\to \GL_n(A)$
	be a deformation to $A$.
	If the image of $\mtr\rho$
	factors through an Artinian local $A'$
	by a local homomorphism $A'\hookrightarrow A$,
	then $\rho$
	is the base change of a deformation to  $A'$.
\end{lem}
\begin{proof}
	Let $k'\subset k$ be the residue fields
	of $A'$ and  $A$,
	by the Carayol lemma proved earlier
	we may assume that  $\rho\mod \rm$
	factors through $\GL_n(k')$.

	Now pick $\{y_1,\cdots,y_{n^2}\}\subset\rho(\Gamma)$
	such that $\bar{y}_i$ form 
	a basis of $M_n(k)$ over $k$ 
	(thus also a basis of $M_n(k')$ over $k'$).
	Since $\rho(\Gamma)$ generates
	$M_n(A)$ over  $A$,
	$\{y_i\}$ is also a basis for $M_n(A)$.
	The same arguments as before then shows that
	$\rho(\Gamma)$ factors through 
	the  $A'$-algebra $R'$ generated by  $\{y_i\}$.
	Since $R'/\fm'R'\cong M_n(k')$ by assumption,
	$R'$ is an Azumaya algebra over the Henselian ring  $A'$.
	Thus $R'$ is isomorphic to a matrix algebra over $A'$,
	and necessarily isomorphic to $M_n(A')$
	by dimension counting.
	The rest of the arguments goes as earlier,
	using the Skolem-Noether theorem
	for the Azumaya algebras.
\end{proof}

Let $R^\square$
be the universal framed deformation.
\[
	 \Hom_k(\fm^\square/(\varpi)+\fm^{\square 2},k)
	 \Longleftrightarrow
	 \Hom(R^\square, k[\epsilon])
	 \Longleftrightarrow
	 D^\square(k[\epsilon])
	 \Longleftrightarrow
	 Z^1(\Gamma,ad\rrho)
\]
Given by  $f\in \Hom_k(\fm^\square/(\varpi)+\fm^{\square 2},k)$ 
gives $\varphi(a+x)=\bar{a}+f(x)\epsilon$ 
for $a\in \oo, x\in \fm^\square$,
well-defined since
$\oo\cap\fm^\square=(\varpi)$.
Given a deformation 
$\rho=\rrho+\theta,\theta\in M_n(k\epsilon)=M_n(k)$,
then  $\theta\rrho^{-1}\in Z^1(\Gamma, ad\rrho)$.

 \[
	 0\to (ad\rrho)^\Gamma\to ad\rrho\to 
	 Z^1(\Gamma,ad\rrho)\to H^1(ad\rrho)\to 0
\]
thus $d=\dim_k Z^1=\dim_k H^1-\dim H^0+n^2$.


Consider $\oo\llbracket x_1,\cdots,x_d\rrbracket\to R^\square$
lifting basis of 
$\fm^\square/(\varpi)+\fm^{\square 2 }$.
Let $J$ be the kernel.

 

$G$ is a profinite group satisfying 
 \[
	 \text{the max pro-p quotient of any finite index subgroup
	 is . f.g.}
	 \Longleftrightarrow
	 \dim_{\mathbb{F}_p}\Hom(G,\mathbb{F}_p)<\infty.
\]
$D^{\square}_{V_{\mathbb{F}_p}}$ is representable,
$D_{V_{\mathbb{F}_p}}$ is representable if $\Hom(V,V)$ is one-dimensional.
Use that  $PGL$ acts freely on framed-deformation.
\begin{itemize}
	 \item (Brauer-Nesbitt) abs.irred over $k$ determined by trace.
	 \item deformation of res.abs.irred determined by trace
		 (Serre Carayol)
	 \item also true if $\dim\Ext^1(\chi_1,\chi_2)=1$
		 for $\chi_1\neq \chi_2$
	 \item If $S\to R$ local homomorphism,
		 and  $tr\rho\in S$,
		 then conjugate to  $\GL_n(S)$
	 \item  $D(\mathbb{F}[\epsilon])\Ext^1(V,V)$
	 \item  $\cong H^1(G, adV)$, 
	 \item and  $\dim D^\square=\dim D+n^2-(ad V)^G$
		 counting fiber.
\end{itemize}
	




\section{Skinner's notes}

For a number field $F$ and a place $v$ of $F$,
let $\rec_{F_v}: F_v^{\times} \rightarrow G_{F_v}^{ab}$ 
be the reciprocity map of local class field theory, 
normalized so that uniformizers map to lifts of the arithmetic Frobenius. 
Similarly, we let $\rec_F: F^{\times} \backslash \mathbb{A}_F^{\times} \rightarrow G_F^{ab}$ 
be the reciprocity map of global class field theory, normalized so that 
$\rec_F\vert_{F_v}=\rec_{F_v}$.
Let $\epsilon:G_\Q\to \Zp^\times$ be the $p$-adic cyclotomic character.

\subsection{Selmer group}
Let $E$ be an elliptic curve over $F$,
Kummer map
\[
    E(F)/mE(F)\xhookrightarrow{\kappa}
    H^1(G_{F,\Sigma},E[m])\coloneqq
    \ker\left\{H^1(F, E[m]) 
    \xrightarrow{\res} \prod_{v \notin \Sigma} H^1\left(I_v, E[m]\right)\right\},
\]
$\Sigma$ such that $v\notin \Sigma$ implies
that $v$ is a finite place not dividing $m$ and $E$ has good reduction at $v$.
Then use $L=F(E[m])\subset F_\Sigma$
\[
    0 \rightarrow H^1\left(\Gal(L / F), E[m]^{G_L}\right) \rightarrow H^1\left(G_{F, \Sigma}, E[m]\right) \xrightarrow{\res} \Hom\left(\Gal(\left(F_{\Sigma} / L\right), E[m]\right).
\]
to show weak Mordell-Weil.

For each place $v$ of $F$ the inclusion of $F$ in the completion $F_v$ induces a commutative diagram
where $\kappa_v$ is just the Kummer map for $E$ over $F_v$. The Selmer group $\Sel_m(E / F)$ for the multiplication by $m$ on $E$ is
\[
\Sel_m(E / F)=\left\{c \in H^1(F, E[m]): \res_v(c) \in \operatorname{im}\left(\kappa_v\right) \forall v\right\} .
\]

This clearly contains the image of $\kappa$. Furthermore, by the argument explained above, if $v$ does not divide $m$ and $E$ has good reduction at $v$ then $\operatorname{im}\left(\kappa_v\right) \subset \ker\left\{H^1\left(F_v, E[m]\right) \xrightarrow{\res} H^1\left(I_v, E[m]\right)\right\}$. In particular, $\Sel_m(E / F) \subseteq H^1\left(G_{F, \Sigma}, E[m]\right)$

\[
\operatorname{Sel}_m(E / F)=\operatorname{ker}\left\{H^1(F, E[m]) \xrightarrow{r e s} \prod_v H^1\left(F_v, E\right)\right\} .
\]

\[
0 \rightarrow E(F) / m E(F) \xrightarrow{\kappa} H^1(F, E[m]) \rightarrow H^1(F, E)[m] \rightarrow 0
\]
and
\[
0 \rightarrow E\left(F_v\right) / m E\left(F_v\right) \xrightarrow{\kappa_u} H^1\left(F_v, E[m]\right) \rightarrow H^1\left(F_v, E\right)[m] \rightarrow 0
\]

And we see that the image of $\kappa$ is just $\operatorname{ker}\left\{\operatorname{Sel}_m(E / F) \rightarrow H^1(F, E)\right\}$. In particular, there is a fundamental exact sequence
\[
0 \rightarrow E(F) / m E(F) \xrightarrow{\kappa} \operatorname{Sel}_m(E / F) \rightarrow \amalg(E / F)[m] \rightarrow 0,
\]
where
\[
\amalg(E / F)=\operatorname{ker}\left\{H^1(F, E) \xrightarrow{r e s} \prod_v H^1\left(F_v, E\right)\right\}
\]

If $m \mid m^{\prime}$ then the inclusion $E[m] \subset E\left[m^{\prime}\right]$ induces a surjection $H^1(F, E[m]) \rightarrow H^1\left(F, E\left[m^{\prime}\right]\right)[m]$ and so a surjection $\operatorname{Sel}_m(E / F) \rightarrow \operatorname{Sel}_{m^{\prime}}(E / F)[m]$. The kernel is just $E\left[\frac{m^{\prime}}{m}\right](F) / m E\left[m^{\prime}\right](F)$. If $F^{\prime} / F$ is a finite extension, then the restriction map $H^1(F, E[m]) \rightarrow H^1\left(F^{\prime}, E[m]\right)$ induces a homomorphism $\operatorname{Sel}_m(E / F) \rightarrow \operatorname{Sel}_m\left(E / F^{\prime}\right)$. Furthermore, if $F^{\prime} / F$ is a Galois extension, then the action of $\operatorname{Gal}\left(F^{\prime} / F\right)$ on $H^1\left(F^{\prime}, E[m]\right)$ defines an action on $\operatorname{Sel}_m\left(E / F^{\prime}\right)$, and the maximal $\operatorname{Gal}\left(F^{\prime} / F\right)$-fixed subgroup contains the image of $\operatorname{Sel}_m(E / F)$.

2.1.4. The $p^{\infty}$-Selmer group. The $p^{\infty}$-Selmer group of $E$ is obtained by taking the direct limits over $n$ of the $p$-power Selmer groups $\operatorname{Sel}_{p^n}(E / F)$ :
\[
\operatorname{Sel}_{p^{\infty}}(E / F)=\underset{n}{\lim } \operatorname{Sel}_{p^n}(E / F) .
\]

Since $\lim _{\longrightarrow} H^1\left(F, E\left[p^n\right]\right)=H^1\left(F, E\left[p^{\infty}\right]\right)$ the $p^{\infty}$-Selmer group can also be directly defined as
\[
\operatorname{Sel}_{p^{\infty}}(E / F)=\operatorname{ker}\left\{H^1\left(F, E\left[p^{\infty}\right]\right) \xrightarrow{r e s} \prod_v H^1\left(F_v, E\right)\right\} .
\]

The natural surjection $H^1\left(F, E\left[p^n\right]\right) \rightarrow H^1\left(F, E\left[p^{\infty}\right]\right)\left[p^n\right]$ induces a surjection $\operatorname{Sel}_{p^n}(E / F) \rightarrow$ $\operatorname{Sel}_{p^{\infty}}(E / F)\left[p^n\right]$ with kernel $E\left[p^{\infty}\right](F) / p^n E\left[p^{\infty}\right](F)$.

Taking the direct limit over the fundamental exact sequences for the multiplication by $p^n$ maps yields the fundamental exact sequence for the $p^{\infty}$-Selmer groups:
\[
0 \rightarrow E(F) \otimes \mathbb{Q}_p / \mathbb{Z}_p \rightarrow \operatorname{Sel}_{p^{\infty}}(E / F) \rightarrow \amalg(E / F)\left[p^{\infty}\right] \rightarrow 0 .
\]

\section{Elliptic curves with CM}

\[
\begin{tikzpicture}
	\draw(0,3)--(0,-2) (-4,0)--(3,0);
	\draw[dashed](-2,-2)--(3,3) (1,-2)--(-3,2);
	\path[pattern=north east lines] 
		(-3,0)--(-3,2)--(-1,2)--(-1,0)--cycle;
	\path[pattern=north east lines] 
		(0,-1)--(0,-2)--(2,-2)--(2,-1)--cycle;
\end{tikzpicture}
\]



\section{L-invariant}
Let $E/\Q$ be an elliptic curve, assume that
\begin{enumerate}[label=(\alph*)]
	\item $E$ has stable reduction modulo $p$
	\item  $E$ has split multiplicative reduction at  $p$.
\end{enumerate}
Tate's $p$-adic uniformization thoery, $q_E\in p\Zp$
such that  $E(\bar{\Q}_p)\cong \bar{\Q}_p^\times/q_E^{\Z}$, 
which is defined over $\Qp$. 
\[
	 \mathcal{L}_p(E)\coloneqq \frac{\log_p(q_E)}{\ordp(q_E)}
\]

\begin{thm}
	If $p\geq 5$, then
	 \[
		 L'_p(E,1)=\mathcal{L}_p(E)\cdot 
		 \frac{L_\infty(E,1)}{\Omega_\infty}
	\]
\end{thm}

Under the assumption that $E$ has split multiplicative reduction
\[
	L_p(E,2-s)=w_p \langle N\rangle^{s-1}L_p(E,s)
\]
where $w_p=-w_\infty$.
\begin{itemize}
	\item if $w_\infty=-1$, then trivial since
		$L'_p(E,1)=0$ then.
	\item if  $w_\infty=1$,  
		then $L_p(E,s)$ has odd order at $s=1$.
		The order is one iff
		$L_\infty(E,1)\neq 0$ and  $\log_p(q_E)\neq 0$.
\end{itemize}
Conjecturally
\[
	ord_{s=1}(L_p(E,s))=
	ord_{s=1}(L_\infty(E,s))=+1
\]

As an example, let $E=X_0(11)$ and  $p=11$.
Let 
 \[
	 \rho_E\colon G_\Q\to \Aut(T_pE),\quad
	 f_E=q\prod(1-q^n)^2(1-q^{11n})^2 
	 \text{ is ordinary at }p.
\]
The universal ordinary Hecke algebra is 
$\Lambda=\Zp\llbracket 1+p\Zp\rrbracket$.
By Hida's theory, there exists a free rank two $\Lambda$-module
$\mathbf{T}$ and 
\[
	\rho\colon G_\Q\to \Aut_\Lambda(\mathbf{T}),\quad
	\mathbf{T}/P_0\mathbf{T}\cong T_pE
\]
where $P_0\subset\Lambda$ is the augmentation ideal.
 \begin{enumerate}[label=(\alph*)]
	\item for $k\in \Zp$,
		let $\sigma_{k-2}\colon\Lambda\to\Zp$
		extends $t\mapsto t^{k-2}$ on $1+p\Zp$.
		Let  $P_k=\ker(\sigma_{k-2})$
	\item $\rho_2=\rho_E$
	\item for each integer  $k\geq 2$,
		there is a new form  $f_k$
		of conductor dividing  $p$
		whose Galois representation is 
		$\mathbf{T}_k=\mathbf{T}/P_k\mathbf{T}$
		 \[
			\text{conductor}=1 \Longleftrightarrow
			k>2 \text{ and } k\equiv 2\mod p-1
		\]
		Otherwise, the condcutor is $p$
		and the Nebentypus is  $\omega^{2-k}$.
	\item 
		\[
			\rho\vert_{G_{\Qp}}\sim 
			\smat
			{\chi\varphi^{-1} & * \\
			0 & \varphi }
		\]
		where $\varphi\colon G_{\Qp}\to \Lambda^\times$
		is unramified and $\chi\colon G_{\Qp}\to \Lambda^\times$
		is such that 
		$\sigma_{k-2}\circ \chi=\chi_0^{k-2}\omega^{2-k}$ 
		where $\chi_0$ is the cyclotomic character.
\end{enumerate}
Let $a_p=\varphi(\Frob_p)\in \Lambda^\times$
The $p$-factor of  $L_\infty(f_k,s)$ is
 \[
	 [(1-\alpha_kp^{-s})(1-\beta_kp^{-s})]^{-1},\quad
	 \alpha_k=a_p(k)\coloneqq \sigma_{k-2}(a_p),
	 \beta_k=
	 \begin{cases}
		 p^{k-1}/\alpha_k, & k>2\text{ and }k\equiv 2\mod p-1,\\
		 0, & \text{otherwise}.
	 \end{cases}
\]
Then  $\alpha_k\equiv \alpha_2=1$ (interpolated by $a_p$ )
and  $\beta_k\equiv 0\mod p$ (cannot be interpolated).
\[
	\mathbf{f}\colon \sum a_nq^n\in \Lambda\llbracket q\rrbracket
	\Longrightarrow
	\mathbf{f}_k=\sum a_n(k)q^n= f_k^*\coloneqq
	f_k(z)-\beta_kf_k(pz)
\]

There is a two variable $p$-adic L-function $L_p(k,s), k,s\in \Zp$.
 \begin{enumerate}[label=(\alph*)]
	 \item $L_p(k,s)$ is analytic for  $k,s\in \Zp$
	 \item  $L_p(2,s)=L_p(E,s)$ and 
		 $L_p(k,s)$ is  $L_p(f_k,s)$.
		 \[
			 L_p(k,s)=\Omega_k\cdot L_p(f_k,s),\Omega_k\in \Qp
			 \text{ and } \Omega_2=1
		 \]
	 \item  $L_p(k,k-s)=-L_p(k,s)$
	 \item  $L_p(k,1)=(1-q_p(k)^{-1})L_p^*(k,1)$
		 where  $L_p^*(k,1)$ is a  $p$-adic analytic function 
		 such that 
		 \[
			 L^*_p(2,1)=\frac{L_\infty(E,1)}{\Omega_E}
		 \]
\end{enumerate}
 For $k\geq2, 0<s_0<k, s\equiv 1\mod p-1$,
  \[
	  L_p(f_k,s_0)=
	  (1-\beta_kp^{-s_0})(1-\alpha_k^{-1}p^{s_0-1})
	  \cdot \frac{L_\infty(f_k,s_0)}{\Omega_{f_k}}=
	  \frac{(1-\alpha_k^{-1}p^{s_0-1})}{(1-\alpha_kp^{-s_0})}
	  \cdot \frac{L^{(p)}_\infty(f_k,s_0)}{\Omega_{f_k}}=
 \]
When $s_0=1$, the factor 
$1-a_p(k)^{-1}$ can be interpolated and is a non-unit,
so $L_p(k,1)$ is divisible by which, the quotient 
$L_p^*(k,1)$ is 
\[
	  L_p(f_k,s_0)=
	  (1-\beta_kp^{-s_0})
	  \cdot \frac{L_\infty(f_k,s_0)}{\Omega_{f_k}}=
\]
When $k-2$, reduces to
$L_p^*(2,1)=\frac{L_\infty(E,1)}{\Omega_E}$ since 
$\beta_2=0,\Omega_{f_2}=\Omega_E, \Omega_2=1$.

By functional equation $L_p(k,k/2)=0$,
at  $(k,s)=(2,1)$,
 \[
	 L_p(k,s)\sim c\cdot (-\frac{1}{2}(k-2)+(s-1))\quad c\in \Zp
\]
Comparing and get 
\[
	L'_p(E,1)=-2a'_p(2)\cdot 
	\frac{L_\infty(E,1)}{\Omega_E}.
\]
It remains to prove $\mathcal{L}_p(E)=-2a'_p(2)$,
recall that  $a_p(2)=0$.

From the uniformization, let $V=V_p(E)$
 \[
	 0\to \Qp(1)\to V\to \Qp\to 0
\]
determines a class $\xi\in H^1(G_{\Qp},\Qp(1))$.
\[
	(\ordp,\log_p)\colon H^1(\Qp(1))\cong \Qp^2
\]
so $\mathcal{L}_p(E)$ is the slope of the line  $\xi$.

 \[
	 0\to \Lambda(\chi\psi)\to \mathbf{T}(\varphi^{-1})\to \Lambda\to 0
	 \Longrightarrow
	 0\to \Qp(\chi_k\psi_k)\to \mathbf{V}_k\to \Qp\to 0
	 \Longrightarrow
	 \xi_k\in H^1(\Qp(\chi_k\psi_k))
\]
where $\psi=\varphi^{-2}$.
Clearly(?) $\xi_k\neq 0$ if  $k$ is close to  $2$,
but $H^1(\Qp(\chi_k\psi_k)$ is one-dimensional  when $k\neq 2$ 
\[
	\frac{d\psi_k(\Frob_p)}{dk}\vert_{k=2}=\mathcal{L}_p(E)
\]

In general, 
start with newform of weight $2$ over  $\Gamma_1(Np)(p\nmid N)$
with split multiplicative reduction at $p(a_p(f)=1)$. 


%\bibliographystyle{amsalpha}
%\bibliography{biblio}
\end{document}
